\documentclass[12pt]{article} %Styl dokumentu
\usepackage{times} %Czcionka
\usepackage[a4paper,left=3.5cm,right=2.5cm,top=3cm,bottom=3cm,includefoot=false,includehead=false]{geometry} %Ustawienia marginesów
\linespread{1} %Interlinia
\setlength{\parindent}{0.5cm} %Wcięcie na początku akapitu
\setlength{\parskip}{1ex plus 0.5ex minus 0.2ex} %Odstępy pomiędzy akapitami

%Dodatkowe pakiety ******************************************************************
\usepackage[T1]{fontenc} %Styl tytułów rozdziałów
%\usepackage{subfig} NIE WIEM DO CZEGO ALE W PRZEJŚCIÓWCE TO MIAŁEM
%\usepackage{float} COS Z FLOATAMI ALE NIE WIEM CO, MOŻE NIE BĘDZIE NAM POTRZEBNE
\usepackage{amsfonts} %Niektóre symbole matematyczne
\usepackage{fixltx2e} %subscript
%\usepackage{pdfpages} %Dołączanie pdfów do tekstu
\usepackage{listings} %Pakiet od listingów programów
\usepackage{xcolor} %Pakiet kolorów do listingów programów Arduino
\usepackage{amsmath} %Pakiet matematyczny
\usepackage{bm,array} %Pakiet do tabel
\usepackage{fancyhdr} %Nagłówek i stopka
\usepackage{graphicx} %Wykresy i obrazy
\usepackage{subfigure} %Dodatkowa biblioteka do obrazów
\usepackage{polski} %Ustawienie języka polskiego
\usepackage[utf8]{inputenc} %Ustawienie kodowania polskich znaków
\usepackage{hyperref} %hiperłącza
\usepackage[labelfont=it,textfont={it}]{caption} %Formatowanie podpisów tabel i rysunków
\usepackage{wrapfig} %tekst obok rysunków
\usepackage{multirow}

%Definiowanie stylu Arduino dla listings*********************************************
\usepackage{xcolor}
\definecolor{dkgreen}{rgb}{0,0.6,0}
\definecolor{dred}{rgb}{0.545,0,0}
\definecolor{dblue}{rgb}{0,0,0.545}
\definecolor{lgrey}{rgb}{0.95,0.95,0.95}
\definecolor{gray}{rgb}{0.4,0.4,0.4}
\definecolor{darkblue}{rgb}{0.0,0.0,0.6}
\definecolor{ArdOr}{rgb}{1,0.451,0.0}
\lstdefinelanguage{Arduino}{
      backgroundcolor=\color{lgrey},  
      basicstyle=\footnotesize \ttfamily \color{black} \bfseries,   
      breakatwhitespace=false,       
      breaklines=true,               
      captionpos=b,                   
      commentstyle=\color{dkgreen},   
      deletekeywords={...},          
      escapeinside={\%*}{*)},                  
      frame=single,                  
      language=C++,                
      keywordstyle=\color{ArdOr},  
      morekeywords={BRIEFDescriptorConfig,string,TiXmlNode,DetectorDescriptorConfigContainer,istringstream,cerr,exit}, 
      identifierstyle=\color{black},
      stringstyle=\color{blue},      
      numbers=left,                 
      numbersep=5pt,                  
      numberstyle=\color{black}, 
      rulecolor=\color{black},        
      showspaces=false,               
      showstringspaces=false,        
      showtabs=false,                
      stepnumber=1,                   
      tabsize=5,                     
      title=\lstname,                 
    }

%Nadpisywanie komend****************************************************************
\renewcommand{\theequation}{\thesubsection.\arabic{equation}}
\numberwithin{equation}{subsection}
\renewcommand{\thefigure}{\thesection.\arabic{figure}}
\renewcommand{\figurename}{Rys.} 
\numberwithin{figure}{section}
\renewcommand{\thetable}{\thesection.\arabic{table}}
\numberwithin{table}{section}
\renewcommand{\captionfont}{\small}
\renewcommand{\lstlisting}{\thesubsection.\arabic{lstlisting}}
\let\stdsection\subsection


\begin{document}

%---------------------------------------------------STRONA TYTUŁOWA-------------------------------------------------------------

\begin{center}
	\large{INSTYTUT AUTOMATYKI\\I INŻYNIERII INFORMATYCZNEJ\\WYDZIAŁ ELEKTRYCZNY\\POLITECHNIKA POZNAŃSKA\\}
\end{center}
\vspace{2cm} 
\begin{center}
	INŻYNIERSKA PRACA DYPLOMOWA\\
\end{center}
\begin{center}
	\Large{\textbf{SYSTEM WSPOMAGAJĄCY KIEROWANIE POJAZDEM\\Z UŻYCIEM ZŁĄCZA DIAGNOSTYCZNEGO}}
\end{center}
\begin{center}
	\large{\textbf{Mateusz BARTOSZ}}
\end{center}
\vspace{4cm}
\begin{flushright}
	Promotor:\\\textbf{Dr inż. Konrad URBAŃSKI}\\
\end{flushright}
\vspace{4cm} 
\begin{center}
	Poznań 2017
\end{center}

%---------------------------------------------------POCZĄTEK DOKUMENTU-------------------------------------------------------------

\thispagestyle{empty}
\newpage
%
%\pagestyle{fancy}
%\rhead{\thepage}
%\lhead{\slshape \rightmark}
%\lfoot{}
%\cfoot{}
%\rfoot{}
%
%\begin{flushright}
%\vspace{4cm}
%\textit{\large{
%\\ 
%\vspace{17cm}
%Składam serdeczne podziękowania\\
%Panu dr. inż. Konradowi Urbańskiemu\\ 
%za cierpliwość, poświęcony czas\\ 
%oraz bezcenne uwagi merytoryczne.}}
%\end{flushright}
%\newpage
\tableofcontents
\thispagestyle{empty}

%---------------------------------------------------STRESZCZENIE-------------------------------------------------------------
\newpage
\thispagestyle{empty}
\section*{Streszczenie}
\vspace{0.5cm}
\hspace{0.5cm}
\newpage

\section*{Abstract}
\thispagestyle{empty}
\vspace{0.5cm}
\hspace{0.5cm}

\newpage

%---------------------------------------------------WSTĘP-------------------------------------------------------------
\section{Wstęp}

	\subsection{Wybór tematu}
		\hspace{0.5cm}Głównym czynnikiem decydującym o wyborze tematu było zainteresowanie rozwiązaniami elektronicznymi stosowanymi we współczesnej motoryzacji oraz chęć podjęcia próby zbudowania układu opartego o własną koncepcję pracującego jako komputer pokładowy w samochodzie osobowym. Kolejnym czynnikiem było umożliwienie ciągłego badania i kontroli parametrów pracy poszczególnych układów pojazdu, w celu uniknięcia, lub wczesnego wykrycia potencjalnych usterek. Dodatkową motywacją była chęć zbudowania układu, który mógłby być w przyszłości praktycznie wykorzystywany w samochodach niewyposażonych w wbudowany komputer pokładowy. 	
	
	\subsection{Cel i zakres pracy}
		\hspace{0.5cm}Celem pracy było zaprojektowanie oraz wykonanie układu przyłącza do gniazda diagnostycznego w samochodzie osobowym Seat Cordoba III oraz zbudowanie interfejsu użytkownika umożliwiającego wizualizację odczytywanych parametrów, kontrolę wartości granicznych, a także przechowanie ich w celach dalszej diagnostyki. Dodatkowym celem było określenie możliwości wykorzystania odbieranych parametrów do opracowania algorytmów umożliwiających wspomaganie kierującego pojazdem, aby zoptymalizować jazdę, zwiększyć bezpieczeństwo podróży oraz zminimalizować ryzyko wystąpienia usterek. 
	
	\subsection{Założenia i wymagania}
		\hspace{0.5cm}Założeniem pracy jest opracowanie kompleksowego układu umożliwiającego odczytywanie, wizualizację oraz kontrolę parametrów odbieranych przez złącze diagnostyczne w samochodzie osobowym Seat Cordoba III, a także określenie możliwości wykorzystania tych parametrów do opracowania algorytmów wspomagających prowadzenie pojazdu.
	
		\newpage

%---------------------------------------------------ZAGADNIENIA WPROWADZAJĄCE-------------------------------------------------------------
\section{Zagadnienia wprowadzające}
	\hspace{0.5cm}We współczesnej motoryzacji wyraźnie można zauważyć tendencję automatyzacji procesu prowadzenia pojazdu oraz kontroli stanu jego parametrów. W samochodach dostępnych na rynku można spotkać bardzo wiele różnych protokołów komunikacyjnych. Cześć z nich służy do komunikacji pomiędzy urządzeniami wewnętrznymi pojazdu, inne do komunikacji z użytkownikiem, w celu zwiększenia komfortu jazdy, a jeszcze inne wykorzystywane są w diagnostyce stanu poszczególnych układów samochodu. Te ostatnie, wyprowadzone są do gniazda diagnostycznego(ang. On-Board Diagnosdic - OBD). W zależności od producenta, modelu oraz roku produkcji pojazdu do dyspozycji są różne protokoły. Umożliwiają one między innymi odczytywanie aktualnych wskazań niektórych czujników, kontrolę zużywania się elementów eksploatacyjnych czy detekcję błędów silnika. W niniejszym rozdziale omówione zostało złącze diagnostyczne w wersji drugiej(ODB2) wraz z udostępnianymi przez nie protokołami komunikacyjnymi. Dodatkowo w ostatnim podrozdziale znajduje się przegląd narzędzi wykorzystanych podczas realizacji pracy.

	\subsection{Złącze diagnostyczne}
		\hspace{0.5cm}Historia złącza diagnostycznego używanego w motoryzacji sięga końcówki lat sześćdziesiątych dwudziestego wieku. Pierwsze komputery pokładowe wprowadzone zostały w samochodach marki Volkswagen w 1968 roku. Dziesięć lat później za sprawą marki Nissan komputery pokładowe pojawiły się w pojazdach konsumenckich. Kolejnym krokiem było wprowadzenie protokołu ALDL  przez General Motors w 1980 roku. Był to pierwszy standard zbliżony do obecnie stosowanego w gniazdach OBD2. Występował w trzech wersjach: dwunasto, dziesięcio i pięciopinowej. Pierwsze wersje były jednokierunkowe i umożliwiały przesyłanie 160 bodów danych. W późniejszych wersjach wprowadzono dwukierunkową transmisję danych o zwiększono szybkości transmisji do 8192 bodów. W 1991 roku agencja California Air Resources Board zażądała, aby każdy nowy pojazd sprzedawany w Kalifornii posiadał wyprowadzenie diagnostyczne. Stało się to impulsem do opracowania i wprowadzenia standardu OBD-I, choć nazwa ta została wprowadzona dopiero po opracowaniu kolejnego standardu OBD w wersji drugiej. Złącze ODB-I nie zostało ściśle ustandaryzowane i każdy producent samochodów mógł wykonać je w swojej wersji. Główną motywacją do wprowadzenia tego standardu było zachęcenie producentów pojazdów do zaprojektowania systemów kontroli emisji spalin. Często spotykana wersja tego złącza umożliwiała odczytywanie kodów błędów poprzez analizę migania diody znajdującej się przy złączu. Miganie diody reprezentowało odpowiednią liczbę dwucyfrową, która była interpretowana jako odpowiedni kod błędu pojazdu. W 1994 roku na bazie poprawionej i uzupełnionej specyfikacji OBD-I powstał standard OBD w wersji drugiej - ODB-II. Jest to najpopularniejszy, aktualnie używany system diagnostyki samochodowej. Od roku 1996 wszystkie nowe samochody sprzedawane w Stanach Zjednoczonych muszą być wyposażone właśnie w gniazdo ODB w wersji drugiej. W 2001 roku standard OBD-II pod nazwą EOBD został wprowadzony jako obowiązkowy w samochodach benzynowych produkowanych w Unii Europejskiej, a w 2003 roku również w samochodach z silnikami wysokoprężnymi.
		
		\newpage
		
		\subsubsection{Złącze diagnostyczne OBD-II}
			\hspace{0.5cm}OMFG ALE DUŻO NAPISAŁEM, TERAZ OBD2 I JAZDA!
		
		
	
		\newpage	
	
	\subsection{Protokoły komunikacyjne w motoryzacji}
		\hspace{0.5cm}GŁÓWNIE TE Z OBD LIN, ISO9141-2, CAN,
						te nowsze
	
		\newpage	
	
	\subsection{Opis wykorzystanych narzędzi}
		\hspace{0.5cm}
		-java
		-javafx
		-raspberry
		-PCB
	
		\newpage
	
%---------------------------------------------------STRUKTURA PROJEKTU-------------------------------------------------------------	
\section{Struktura projektu}
	\hspace{0.5cm}
	-założenia całościowe
	-schemat
	-opis ogólny	
	
	\newpage	
	
%---------------------------------------------------PROJEKT PŁYTKI-------------------------------------------------------------	
\section{Projekt układu do komunikacji ze złączem diagnostycznym}
	\subsection{Schemat elektryczny}
		\hspace{0.5cm}
	
		\newpage
	
	\subsection{Projekt płytki drukowanej}
		\hspace{0.5cm}
	
		\newpage
%---------------------------------------------------APLIKACJA KOMUNIKACYJNA-------------------------------------------------------------	
	
\section{Aplikacja przetwarzająca dane ze złącza diagnostycznego}
	\hspace{0.5cm}
	
	\newpage
	
%---------------------------------------------------INTERFEJS GRAFICZNY-------------------------------------------------------------	
	
\section{Interfejs użytkownika}
	\hspace{0.5cm}
	
	\newpage
	
\section{Podsumowanie}
	
	\hspace{0.5cm} 
	
	\newpage	
	
\begin{thebibliography}{99}

		\bibitem{Massalski80}
		Massalski J., Massalska M., \emph{Fizyka dla inżynierów część I}, Wydawnictwo Naukowo-Techniczne, Warszawa, 1980.
	
		\bibitem{Halliday12}
		Halliday D., Resnick R., Walker J., \emph{Podstawy fizyki tom 2}, Wydawnictwo Naukowe PWN, Warszawa, 2012.

		\bibitem{Sawieliew98}
		Sawieliew I. W., \emph{Wykład z fizyki tom 1}, Wydawnictwo Naukowe PWN, Warszawa, 2002.
	
		\bibitem{Bogusz10}
		Bogusz W., Grabarczyk J., Krok F., \emph{Podstawy fizyki}, Oficyna Wydawnicza Politechniki Warszawskiej, Warszawa, 2010.
	
		\bibitem{Halliday12}
		Halliday D., Resnick R., Walker J., \emph{Podstawy fizyki tom 1}, Wydawnictwo Naukowe PWN, Warszawa, 2012.
	
		\bibitem{Szuba10}
		Szuba S., \emph{Ćwiczenia laboratoryjne z fizyki}, Poznańska Księgarnia Akademicka, Poznań, 2010.
	
		\bibitem{Szydlowski03}
		Szydłowski H., \emph{Pracownia fizyczna wspomagana komputerem}, Wydawnictwo Naukowe PWN, Warszawa, 2003.
		
		\bibitem{Ostwald12}
		Ostwald M., \emph{Podstawy wytrzymałości materiałów}, Wydawnictwo Politechniki Poznańskiej, Poznań, 2012.
	
		\bibitem{Ostwald12_ZbiorZadan}
		Ostwald M., \emph{Wytrzymałości materiałów. Zbiór zadań}, Wydawnictwo Politechniki Poznańskiej, Poznań, 2012.
		
		\bibitem{Niezgodzinski09}
		Niezgodziński M. E., Niezgodziński T., \emph{Wzory, wykresy i tablice wytrzymałościowe}, Wydawnictwo Naukowe PWN, Warszawa, 2009.
	
		\bibitem{Hadam04}
		Hadam P., \emph{Projektowanie systemów mikroprocesorowych}, Wydawnictwo BTC, Warszawa, 2004.
	
		\bibitem{Ziętek11}
		Ziętek B., \emph{Optoelektronika}, Wydawnictwo Naukowe Uniwersytetu Mikołaja Kopernika, Toruń, 2011.
	
		\bibitem{Francuz11}
		Francuz T., \emph{Język C dla mikrokontrolerów AVR}, Wydawnictwo Helion, Gliwice, 2011.
	
		\bibitem{Anderson14}
		Anderson R., Cervo D., \emph{Arduino dla zaawansowanych}, Wydawnictwo Helion, 2014.
	
		\bibitem{Chruściel08}
		Chruściel M., \emph{LabVIEW w praktyce}, Wydawnictwo BTC, Legionowo, 2008.
	
		\bibitem{Chomicki14}
		Chomicki W., \emph{Magisterska praca dyplomowa}, Politechnika Poznańska, Poznań, 2014.
	
		\bibitem{niusb6009}
		http://www.ni.com/datasheet/pdf/en/ds-218 (28.06.2016)
		
		\bibitem{silnikKrokowy16}
		http://www.silniki.pl/download/57bygh\_032011.pdf (28.06.2016)
		
		\bibitem{sterownikSilnikaKrokowego16}
		http://www.silniki.pl/download/SMC104\_karta\_katalogowa.pdf (28.06.2016)
		
		\bibitem{uln2003a}
		http://www.ti.com/lit/ds/symlink/uln2003a.pdf (28.06.2016)
		
		\bibitem{panasonicHG-C1400}
		https://www.panasonic-electric-works.com/ \\ cps/rde/xbcr/pew\_eu\_en/ds\_hgc\_3219\_en.pdf (28.06.2016)
		
		\bibitem{RysPanasonicHG-C1400}
		https://www.panasonic-electric-works.com/ \\ 
		cps/rde/xbcr/pew\_eu\_en/ds\_hgc\_applications\_en.pdf (28.06.2016)
		
		\bibitem{kamera16}
		http://wiibrew.org/wiki/Wiimote\#IR\_Camera (28.06.2016)
		
		\bibitem{Arduino16}
		https://www.arduino.cc/en/Main/ArduinoBoardUno (28.06.2016)
		
		\bibitem{LabVIEW16}
		http://poland.ni.com/labview (28.06.2016)
		
		\bibitem{ArduinoSoftware16}
		https://www.arduino.cc/en/Main/OldSoftwareReleases\#previous (28.06.2016)
		

\end{thebibliography}
	\newpage

	\listoffigures{}
	\newpage

	\listoftables
	\newpage

\section*{Dokumentacja techniczna}
	\lhead{\textit{DOKUMENTACJA TECHNICZNA}}

	\begin{enumerate}
		\item{Obudowa Arduino Uno}
		\item{Korpus obudowy Arduino Uno}
		\item{Pokrywa obudowy Arduino Uno}
	\end{enumerate}


	\newpage
	\thispagestyle{empty}
	\newgeometry{a4paper,left=2.5cm,right=2.5cm,top=3cm,bottom=2cm,includefoot=false,includehead=false}

	\newpage
	\thispagestyle{empty}
	\newgeometry{a4paper,left=2.5cm,right=2.5cm,top=3cm,bottom=2cm,includefoot=false,includehead=false}
	

	\newpage
	\thispagestyle{empty}
	\newgeometry{a4paper,left=2.5cm,right=2.5cm,top=3cm,bottom=2cm,includefoot=false,includehead=false}
	
	\thispagestyle{empty}
\end{document}